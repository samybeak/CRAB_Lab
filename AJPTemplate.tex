% !TeX encoding = UTF-8
% !TeX spellcheck = en_US
% !TeX TS-program = pdflatex
% !BIB TS-program = bibtex

% Template for American Journal of Physics
% Originally from the AJP website with new
% customizations by Ted Corcovilos (2014) to accommodate BibTeX
% including a custom BibTeX style file (AJP.bst) and hyperlinks

\documentclass[prb,preprint,letterpaper,noeprint,longbibliography,nodoi,footinbib]{revtex4-1} 
% The line above defines the type of LaTeX document.
% Note that AJP uses the same style as Phys. Rev. B (prb).

% The % character begins a comment, which continues to the end of the line.

% BEGIN bibliography customizations
\usepackage[colorlinks, allcolors=blue]{hyperref}
\bibliographystyle{AJP}
% END customizations

\usepackage{amsmath}  % needed for \tfrac, \bmatrix, etc.
\usepackage{amsfonts} % needed for bold Greek, Fraktur, and blackboard bold
\usepackage{graphicx} % needed for figures

\begin{document}

% Be sure to use the \title, \author, \affiliation, and \abstract macros
% to format your title page.  Don't use lower-level macros to  manually
% adjust the fonts and centering.

\title{Title}
% In a long title you can use \\ to force a line break at a certain location.

%\author{Daniel V. Schroeder}
%\email{dschroeder@weber.edu} % optional
%\altaffiliation[permanent address: ]{101 Main Street, 
%  Anytown, USA} % optional second address
%% If there were a second author at the same address, we would put another 
%% \author{} statement here.  Don't combine multiple authors in a single
%% \author statement.
%\affiliation{Department of Physics, Weber State University, Ogden, UT 84408-2508}
%% Please provide a full mailing address here.
%
%\author{David P. Jackson}
%\email{ajp@dickinson.edu}
%\affiliation{Department of Physics, Dickinson College, Carlisle, PA 17013}

% See the REVTeX documentation for more examples of author and affiliation lists.

\date{\today}

\begin{abstract}

Abstract...

\end{abstract}
% AJP requires an abstract for all regular article submissions.
% Abstracts are optional for submissions to the "Notes and Discussions" section.

\maketitle

\section{Introduction}

%\begin{figure}[h!]
%% The bracketed code determines the figure's placement:  "h" stands for 
%% "here", telling LaTeX to put the figure as close to the current location 
%% as possible.  The ! overrides LaTeX's tendency to try to find a location 
%% that it thinks is better.  But don't agonize over the exact figure placement 
%% in your submitted manuscript.  For your initial submission, just make sure 
%% each figure is reasonably close to where it's first referenced.
%\centering
%\includegraphics{fig/GasBulbData.eps}
%\caption{Pressure as a function of temperature for a fixed volume of air.  
%The three data sets are for three different amounts of air in the container. 
%For an ideal gas, the pressure would go to zero at $-273^\circ$C.  (Notice
%that this is a vector graphic, so it can be viewed at any scale without
%seeing pixels.)}
%\label{gasbulbdata}
%\end{figure}
% When an AJP manuscript is conditionally accepted for publication, we
% will request an editable copy of the manuscript in which the figures 
% themselves have been removed, and all figure captions have been moved
% to the end.  You should then comment-out the \includegraphics line
% and move the entire \begin{figure} ... \end{figure} block to the end
% of the source file, just before \end{document}.

% NOTE: place figure files in the `fig' folder and include this in the filename above.

\section{Conclusion}


%\appendix*   % Omit the * if there's more than one appendix.
%
%\section{Uninteresting stuff}
%
%Appendices are for material that is needed for completeness but
%not sufficiently interesting to include in the main body of the paper.  Most
%articles don't need any appendices, but feel free to use them when
%appropriate.  This sample article needs an appendix only to illustrate how 
%to create an appendix.


\begin{acknowledgments}

We gratefully acknowledge...

\end{acknowledgments}

% BEGIN While writing use BibTeX
\bibliography{AJPTemplate} % For BibTex
% END

%BEGIN For final version, copy and paste contents of .bbl file here after running BibTeX (note: biber does not work in this case.)

%\begin{thebibliography}{99}
% The numeral (here 99) in curly braces is nominally the number of entries in
% the bibliography. It's supposed to affect the amount of space around the
% numerical labels, so only the number of digits should matter--and even that
% seems to make no discernible difference.

%\bibitem{noBIBTeX} Many \LaTeX\ users manage their bibliographic data with 
%a tool called BIB\TeX.  Unfortunately, AJP cannot accept BIB\TeX\ files; all 
%bibliographic references must be incorporated into the manuscript file
%as shown here, at least when you send an editable file for production.

%\end{thebibliography}

% If your manuscript is conditionally accepted, the editors will ask you to
% submit your editable LaTeX source file.  Before doing so, you should move
% all tables and figure captions to the end, as shown below.  Tables come 
% first, followed by figure captions (with figure inclusions commented-out).
% Figures should be submitted as separate files, collected with the
% LaTeX file into a single .zip archive.

%\newpage   % Start a new page for tables


\end{document}
